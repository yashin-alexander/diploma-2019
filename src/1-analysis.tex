\mysection{АНАЛИЗ ПРЕДМЕТНОЙ ОБЛАСТИ}
\subsection{Встроенные системы}
\label{subsec:linux-for-embedded-systems}

\textbf{Встроенные вычислительные системы} (ВВС) – специализированные (заказные) вычислительные системы (ВС), непосредственно взаимодействующие с объектом контроля или управления и объединенные с ним единой конструкцией \cite{EMBEDDED}.

В силу специфики использования, обычно встроенные системы имеют низкое энергопотребление, предназначены для выполнения определенного набора задач и не являются универсальной вычислительной платформой, такой как компьютеры. По этим причинам, многие встроенные системы не используют операционную систему. Тогда программист проводит собственную разработку всего программного обеспечения, которое управляет оборудованием, практически не задействовав многозадачность и взаимодействие с пользователем. Программное обеспечение в таком случае работает непосредственно с оборудованием встроенной системы.

Однако в современном мире существуют более сложные встроенные системы, с расширенным списком решаемых задач. 
В такой многозадачной системе, должны быть реализованы следующие механизмы:
\begin{dashitemize}
  \item распределение памяти;
  \item предоставление времени ЦП разным потокам, процессам;
\end{dashitemize}

В таком случае программист может использвать ОС, это подход позволяет сосредоточиться на решении задачи, возложенной на ВВС, делегировав управление аппаратными средствами ОС.

Однако операционные системы очень трудно создавать, и это приводит к увеличению количества строк кода в проекте. 
При этом важно отметить, что влияние на количество ошибок в коде оказывает количество строк, то есть чем больше строк, тем больше ошибок \cite{EMBEDDED}. 
По этой же причине уст­ра­не­ние всех оши­бок в прак­ти­че­ски зна­чи­мом ПО слиш­ком тру­до­ём­ко, вме­сто это­го при его соз­да­нии обыч­но пы­та­ют­ся дос­тичь мак­си­маль­но воз­мож­но­го при за­дан­ных за­тра­тах уров­ня ка­че­ст­ва, как мож­но боль­ше сни­зить ве­ро­ят­ность про­яв­ле­ния оши­бок и ущер­ба от них. 
Так, как операционные системы развиваются и поддерживаются в течение долгого периода времени\cite{TANENBAUM}, то во встроенных решениях часто используют уже существующие ОС.

\newpage
\subsection{Linux во встроенных системах}

Большое распространение во встроенных системах получили ОС на основе Linux.

\textbf{Дистрибутив Linux} - это операционная система, в основе которой лежит ядро Linux и, зачастую, система управления пакетами. 
Дистрибутив может быть представлен в виде предварительно скомпилированных двоичных файлов и пакетов, собранных сопровождающими дистрибутива, или в виде источников в сочетании с инструкциями о том, как их скомпилировать.

Среда разработки в программировании встроенных систем обычно сильно отличается от сред тестирования и производства, они могут использовать разные архитектуры чипов, программные стеки и даже операционные системы.
Во встроенных системах, поскольку аппаратная платформа зачастую выполняется на заказ, разработчик ОС обычно предпочитает генерировать дистрибутив с нуля, из источников.
Системы сборки решают задачу создания образа Linux дистрибутива. 

\newpage
\subsection{Обзор существущих систем сборки}
Как правило, выходные данные сборки состоят из полного образа программного обеспечения для целевого устройства, включая ядро, драйверы устройств, библиотеки, прикладное программное обеспечение и загрузчик.

\newpage
\subsection{Постановка задач исследования}
В рамках данной работы требуется спроектировать и реализовать программное обеспечение для сборки дистрибутивов операционных систем на основе Linux для встроенных систем.
Система сборки должна быть обладать следующими свойствами:
\begin{dashitemize}
  \item кроссплатформенность;
  \item возможность вручную конфигурировать необходимые предустановленные в образ пакеты;
  \item возможность одновременной сборки нескольких образов: минимального, серверного, расширенного;
  \item наличие загрузчика в образах;
  \item наличие пакетного менеджера в образах;
  \item пригодность для использования образов во время разработки ВС;
\end{dashitemize}
\newpage
