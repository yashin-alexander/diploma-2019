\mysection{АНАЛИЗ ПРЕДМЕТНОЙ ОБЛАСТИ}
\subsection{Linux для встроенных систем}
\label{subsec:linux-for-embedded-systems}

Встроенные вычислительные системы (ВВС) – специализированные (заказные) вычислительные системы (ВС), непосредственно взаимодействующие с объектом контроля или управления и объединенные с ним единой конструкцией.

В силу специфики использования, обычно встроенные системы имеют низкое энергопотребление, предназначены для выполнения определенного набора задач и не являются универсальной вычислительной платформой, такой как компьютеры. По этим причинам, многие встроенные системы не используют операционную систему. Тогда программист проводит собственную разработку всего программного обеспечения, которое управляет оборудованием, практически не задействовав многозадачность и взаимодействие с пользователем. Программное обеспечение в таком случае работает непосредственно на оборудовании встроенной системы. \cite{EMBEDDED}
 
Однако в современном мире существуют более сложные встроенные системы, с расширенным списком решаемых задач.
В такой многозадачной системе, должны быть реализованы:
\begin{dashitemize}
  \item механизмы распределения памяти;
  \item механизмы предоставления времени ЦП разным потокам, процессам;
\end{dashitemize}

В качестве готовой операционной системы во встроенных решения зачастую используются системы на основе Linux. 

Дистрибутив Linux - это операционная система, в основе которой лежит ядро Linux и, зачастую, система управления пакетами. Дистрибутив может быть представлен в виде предварительно скомпилированных двоичных файлов и пакетов, собранных сопровождающими дистрибутива, или в виде источников в сочетании с инструкциями о том, как их скомпилировать. Во встроенных системах, поскольку аппаратная платформа зачастую выполняется на заказ, разработчик ОС обычно предпочитает генерировать дистрибутив с нуля, из источников.
Системы сборки решают задачу создания образа Linux дистрибутива.
