\mysection*{АНАЛИЗ ПРЕДМЕТНОЙ ОБЛАСТИ}
\subsection{Linux для встроенных систем}
\label{subsec:linux-for-embedded-systems}


Дистрибутив Linux - это операционная система, созданная из коллекции программного обеспечения, в основе которой лежит ядро Linux и, зачастую, система управления пакетами. Дистрибутив может быть представлен в виде предварительно скомпилированных двоичных файлов и пакетов, собранных сопровождающими дистрибутива, или в виде источников в сочетании с инструкциями о том, как их скомпилировать.

Во встроенных системах, поскольку аппаратная платформа зачастую выполняется на заказ, разработчик ОС обычно предпочитает генерировать дистрибутив с нуля, из источников.

