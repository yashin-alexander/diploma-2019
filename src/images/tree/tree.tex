% !TEX program = xelatex
% !TeX spellcheck = ru_RU_yo

\documentclass[crop,tikz]{standalone}

\usepackage{fontspec}
\setmainfont{PT Astra Sans}

\usepackage{polyglossia}
\setdefaultlanguage[spelling=modern]{russian}
\newfontfamily{\cyrillicfont}{PT Astra Sans}
\newfontfamily{\cyrillicfontsf}{PT Astra Sans}
\newfontfamily{\cyrillicfonttt}{PT Mono}

\usetikzlibrary{calc,arrows.meta}

\definecolor{darkgray}{RGB}{85,85,85}

\newcounter{treeline}

\newcommand{\treeroot}[1]{ % Title
  \node[above] at (0,1) {\texttt{#1}};

  \setcounter{treeline}{0}
}

\newcommand{\treeentry}[2]{ % Title, Level
  \draw[thick]
    (#2 - 1, -\value{treeline} * 2 + 1) --
    (#2 - 1, -\value{treeline} - \value{treeline} - 1) --
    (#2 + 1, -\value{treeline} - \value{treeline} - 1) node[right] {\texttt{#1}};

  \node at (#2 + 1, -\value{treeline} - \value{treeline} - 1) [yshift=-3pt] {\textbullet};

  \foreach \x in {1,...,#2}
  {
    \draw[thick]
      (\x - 1, -\value{treeline} * 2 + 1) --
      (\x - 1, -\value{treeline} - \value{treeline} - 1);
  }

  \stepcounter{treeline}
}

\newcommand{\treeentrycommented}[4]{ % Title, Level, Dummy node, Comment
  \draw[thick]
    (#2 - 1, -\value{treeline} * 2 + 1) --
    (#2 - 1, -\value{treeline} - \value{treeline} - 1) --
    (#2 + 1, -\value{treeline} - \value{treeline} - 1) node[right] (#3) {\texttt{#1}};

  \node at (#2 + 1, -\value{treeline} - \value{treeline} - 1) [yshift=-3pt] {\textbullet};

  \draw[
    gray,
    line width=3pt, line cap=round, dash pattern=on 0pt off 10pt
  ] (#3) -- (20, -\value{treeline} - \value{treeline} - 1) node[darkgray,right] {#4};

  \foreach \x in {1,...,#2}
  {
    \draw[thick]
      (\x - 1, -\value{treeline} * 2 + 1) --
      (\x - 1, -\value{treeline} - \value{treeline} - 1);
  }

  \stepcounter{treeline}
}
\begin{document}

\begin{tikzpicture}
  \fontsize{30}{36}\selectfont

  \treeroot{\textbf{debian-based-image/}}
    \treeentrycommented{\textbf{lib/}}{1}{webpack}{Библиотечные файлы системы сборки}
      \treeentry{*.sh}{2}
      \treeentry{*.sed}{2}
    \treeentrycommented{\textbf{stageN/}}{1}{src}{Исходные файлы стадии}
      \treeentry{N0-step}{2}{assets}{Шаги стадии}
        \treeentry{\textbf{files/}}{3}
          \treeentrycommented{\textbf{...}}{4}{assets}{Файлы, необходимые для выполнения шагов стадии}
        \treeentrycommented{\textbf{/00-run.sh}}{3}{assets}{Подшаги}
        \treeentry{01-run.sh}{3}
        \treeentry{...}{3}
      \treeentry{\textbf{N1-step/}}{2}
      \treeentrycommented{\textbf{prerun.sh}}{3}{assets}{Файлы, выполняемые до выполнения шагов стадии}
      \treeentry{...}{3}
      \treeentry{\textbf{work/}}{2}
        \treeentrycommented{\textbf{stage0/}}{3}{common}{Рабочая директория для страдии 0}
          \treeentrycommented{textbf{rootfs}}{4}{assets}{Содержимое rootfs текущей стадии}
          \treeentry{biuld.log}{4}{Лог сборки}
        \treeentry{...}{3}
      \treeentrycommented{\textbf{deploy/}}{1}{assets}{Собранные файлы образов}
      \treeentrycommented{\textbf{Dockerfile}}{1}{assets}{Конфигурационный файл Docker}
      \treeentrycommented{\textbf{build.sh}}{1}{assets}{Исполняемый файл системы}
      \treeentrycommented{\textbf{config}}{1}{assets}{Конфигурационный файл системы сборки}
\end{tikzpicture}
\end{document}
