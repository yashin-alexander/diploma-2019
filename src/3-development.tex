\mysection{ОПИСАНИЕ РЕАЛИЗАЦИИ ПРОГРАММНОГО ОБЕСПЕЧЕНИЯ}
\subsection{Подготовка окружения для разработки}
Основным средсвом для реализации системы сборки является командная оболочка bash.
Система также основана на утилитах debootstrap и chroot.

\textbf{Bash} - это командный процессор, соответвующий стандарту POSIX и предоставляющий пользователям интерфейс для взаимодействия с другими программами.
Bash также обладает возможностью читать команды из файлов скриптов, состоящих из списка команд.
Как и многие другие оболочки, в bash поддерживается автодополнение имён дерева каталогов.
Также в bash поддерживаются переменные, существуют стандартные конструкции для ветвления, циклов, объявления функций и т. п..
Многие лексические особенности были заимствованы из оболочки sh. Кроме того, некоторые функции были заимствованы из csh и ksh.

\textbf{Debootstrap} - инструмент, позволяющий установить базовую систему Debian в поддиректорию уже существующей системы.
Для этого не требуется установочный компакт-диск, необходимо лишь обеспечить доступность к репозитория Debian.
Его также можно установить и запустить из другой ОС, поэтому, например, существует возможность использовать debootstrap для инсталяции Debian на свободный раздел работающей системы Gentoo.
Инструмент также может использоваться при создании rootfs для машины с другой архитектурой, которая называется «перекрестная перезагрузка»\cite{DEBOOTSTRAP}.

\textbf{Chroot} - Unix системах это операция, которая изменяет корневой каталог для текущего запущенного процесса и всех его дочерних процессов.
Программа, которая запускается в такой измененной среде, не имеет доступа к файлам за пределами указанного дерева каталогов.

\newpage
\subsection{Контейнеризация}
Для дострижения требования кроссплатформенности предлагается обернуть систему сборки в контейнер.
Контейнеры - это способ инкапсуляции приложения с его зависимостями.
На первый взгляд, они кажутся просто облегченной формой виртуальных машин (ВМ) - подобно виртуальной машине, контейнер содержит изолированный экземпляр операционной системы (ОС), который можно использовать для запуска приложений \cite{DOCKER}.
Однако контейнеры имеют ряд преимуществ, которые позволяют использовать их в задачах, которые являются сложными или невозможными для традиционных виртуальных машин:

\begin{itemize}
  \item Контейнеры совместно используют ресурсы с операционной системой хоста, что делает их на порядок более эффективными.
  \item Приложения, работающие в контейнерах, требуют минимальных или нулевых накладных расходов по сравнению с приложениями, работающими на основной операционной системе.
  \item Портативность контейнеров может устранить целый класс ошибок, которые могут вызываться небольшими изменениями в рабочей среде.
  \item Пользователи могут загружать и запускать сложные приложения без необходимости тратить часы на конфигурацию и установку.
\end{itemize}
Для разрабатываемой системы сборки виртуализации на уровне операционной системы необходима по причине использования небезопасных вызовов chroot с динамически определяемыми аргументами.
Это может повлечь нарушение целостности хост-системы.
В качестве средства управления контейнерами был выбран docker как один из самых распространенных и известных автору.
Для менеджера контейнеров был написан конфигурационный файл, создающий минимально необходимое для сборки окружение.

\newpage
\subsection{Структура проекта}
Сборка образов разделена на несколько стадий, а стадии на несколько шагов для логической ясности и модульности.
Структура системы подробно описана на рисунке \ref{fig:project-tree}.
\begin{figure}[h!]
  \centering
  \setlength{\fboxsep}{5pt}
  \includegraphics[width=.95\textwidth]{build/images/tree/tree}
  \vspace*{6pt}
  \caption{Структура проекта}\label{fig:project-tree}
\end{figure}


\newpage
\subsection{Стадии сборки}
\subsubsection{Создание файловой системы}
Основная цель этого этапа - создать удобную файловую систему.
Для генерации базовой системы при создании системы объявляется функция bootstrap.
В случае, если арихтектура операционной системы-хоста arm64, используется классическая утилита debootstrap. В противном случае используется оболочка debootstrap для qemu.
\lstinputlisting[
  caption={Реализация функции bootstrap},
  label={lst:pytest__short}
]{source/debootstrap.sh}

После получения базовой файловой системы, необходимо добавить конфигурирование пакетного менеджера, установку пакетов локалей.
Кроме того, необходимо выполнить деинсталяцию окружения рабочего стола gnome, так как пакеты gnome являются достаточно объемными и нет необходимости включать их в результирующий образ.
Для работы с .deb пакетами в базовой файловой систем используется утилита chroot, для проведения манипуляций с пакетами внутри базовой файловой системы используется пакетный менеджер apt.

\lstinputlisting[
  caption={Деинсталяция окружения gnome},
  label={lst:pytest__short}
]{source/remove_gnome.sh}

\subsubsection{Кросс-компиляция зависимостей}
На этом этапе накладываются патчи и компилируются важные зависимости для будущего образа, такие как Linux, u-boot, ARM Trusted Firmware и другие.
Для обеспечения кросс-компиляции в системе сборки используется набор утилит для компиляции и отладки программ компании Linaro.
\lstinputlisting[
  caption={Загрузка утилит компиляции Linaro},
  label={lst:pytest__short}
]{source/linaro.sh}

После выполнения этого этапа результирующий образ системы способен загружаться, настраивается загрузчик, обеспечивается работоспособность сети.
На этом этапе результирующая система может загрузиться с консоли, в которой у пользователя есть средства для выполнения основных задач, необходимых для настройки и использования дистрибутива.
Эта система настолько минимальна, насколько это возможно, и ее пока нельзя реально использовать в традиционном смысле, так как все еще остается необходимость добавить ряд пакетов, необходимых для разработки и отладки встроенных систем.

\subsubsection {Создание серверного образа}
Система устанавливает некоторые инструменты разработки, добавляет поддержку Wi-Fi и Bluetooth и другие основные пакеты для управления оборудованием.
Происходит инсталяция таких пакетов как ssh, python3, i2c-tools, alsa-utils и других.
Кроме того, на этом этапе накладывается патч на конфигурационный файл демона sshd, добавляющий возможность доступа по ssh для суперпользователя.
Наложение патчей происходит с использованием функции apply\_patches из файла lib/common.sh.
\lstinputlisting[
  caption={Реализация функции apply\_patches},
  label={lst:pytest__short}
]{source/apply-patches.sh}
\newpage

\subsubsection {Экспорт образов}
На этом этапе система размечает файл образа и создает файловую систему.
После этого происходит упаковка и установка скомпилировнных на предыдущих шагах пакетов.
\lstinputlisting[
  caption={Реализация функций упаковки и установки .deb пакетов},
  label={lst:pytest__short}
]{source/debpack.sh}
После окончания установки пакетов на этой стадии созданный файл образа маркируется датой создания и перемещается в директорию доступную для дальнейшей выгрузки.
\newpage
\subsection{Сборка образов с различными конфигурациями}
В зависимости от заданных пользователем параметров, в процесс создания образа могут быть включены дополнительные этапы, что может повлечь сборку нескольких образов: минимального, серверного и полного.
\subsubsection{Минимальный образ}
Этот образ содержит минимальную файловую систему для дальнейших манипуляций с ней. Не установлено никаких дополнительных пакетов.
Это достигается в основном за счет использования debootstrap, который создает минимальную файловую систему, подходящую для использования в качестве основы rootfs в системах Debian.
\subsubsection{Серверный образ}
Этот образ расширяет количество предустановленных пакетов, по умолчанию добавляются пакеты ssh, vim, python3, parted и другие.
Он также обладает сконфигурированными польщовательскими группами и предоставляет пользователю доступ к sudo и стандартным группам полномочий.
\subsubsection{Полный образ}
Полный образ является расширенным серверным образом и включает в себя среду рабочего стола xfce4.
\newpage
