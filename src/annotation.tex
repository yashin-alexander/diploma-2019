\newgeometry{
  top=20mm,
  right=15mm,
  bottom=20mm,
  left=20mm,
  bindingoffset=0cm
}

\thispagestyle{empty}

\begin{center}
  {
    \bfseries
    {
      \subnormal
      Министерство науки и высшего образования Российской Федерации
    } \\[-0.5em]
    {
      \scriptsize
      ФЕДЕРАЛЬНОЕ ГОСУДАРСТВЕННОЕ АВТОНОМНОЕ ОБРАЗОВАТЕЛЬНОЕ УЧРЕЖДЕНИЕ ВЫСШЕГО ОБРАЗОВАНИЯ
    } \\[-0.25em]
    {
      \subnormal
      “САНКТ-ПЕТЕРБУРГСКИЙ НАЦИОНАЛЬНЫЙ ИССЛЕДОВАТЕЛЬСКИЙ \\[-0.5em]
      УНИВЕРСИТЕТ ИНФОРМАЦИОННЫХ ТЕХНОЛОГИЙ, \\[-0.75em]
      МЕХАНИКИ И ОПТИКИ”
    }
  }
\end{center}

\small

\begin{center}
  \vskip -1em
  {
    \bfseries
    {
      \large
      АННОТАЦИЯ \\
    }
    ВЫПУСКНОЙ  КВАЛИФИКАЦИОННОЙ  РАБОТЫ \\[1.5em]
  }
\end{center}

{
  \parindent0pt

  \titledline{\textbf{Студент}}
  $\underset{
    \text{\scriptsize (Фамилия, Имя, Отчество)}
  }{
    \underline{\makebox[\remaining][s]{~Яшин Александр Павлович\hfill}}
  }$ \\[-0.5em]

  \textbf{Наименование темы ВКР}
  \uline{~Разработка системы сборки Embedded Linux \hfill} \\[-1em]

  \textbf{Наименование организации, где выполнена ВКР}
  \uline{~Университет ИТМО\hfill} \\[-1.75em]
}

\begin{center}
  \textbf{ХАРАКТЕРИСТИКА ВЫПУСКНОЙ КВАЛИФИКАЦИОННОЙ РАБОТЫ}
\end{center}

\vskip -1em

{
  \parindent0pt

  \textbf{1 Цель исследования}
  \uline{Проектирование и реализация системы сборки Linux дистрибутивов для встроенных систем.\hfill} \\[-1em]

  \textbf{2 Задачи, решаемые в ВКР} \\
  \uline{
    2.1 Провести анализ существующих решений, реализующих сборку Linux для встраиваемых\ \ \ \ \ \ \      \newline устройств;\hfill
  }\\
  \uline{
    2.2 Определить требования к системе сборки;\hfill
  }\\
  \uline{
    2.3 Спроектировать архитектуру системы сборки;\hfill
  }\\
  \uline{
    2.4  Реализовать программное обеспечение системы сборки в соответствии с разработанной ~архи\-тектурой;\hfill
  }\\
  \uline{
    2.5 Произвести тестирование системы.\hfill
  }\\[-1em]

  \textbf{3 Число источников, использованных при составлении обзора}
  \uline{\hfill 5\hfill} \\[-1em]

  \textbf{4 Полное число источников, использованных в работе}
  \uline{\hfill 11\hfill} \\[-1em]

  \textbf{5 В том числе источников по годам}
  \begin{figure}[h!]
    \centering
    \begin{tabular}{| *{6}{>{\centering\small\vspace{2pt}}m{2cm} |}}
      \toprule
      \multicolumn{3}{|>{\bfseries\small}c|}{Отечественных} & \multicolumn{3}{>{\bfseries\small}c|}{Иностранных} \tabularnewline
      \midrule
      Последние 5 лет & От 5 до 10 лет & Более 10 лет & Последние 5 лет & От 5 до 10 лет & Более 10 лет \tabularnewline
      \midrule
      0 & 1 & 0 & 10 & 0 & 0 \tabularnewline
      \bottomrule
    \end{tabular}
  \end{figure}\\[-2.5em]

  \titledline{\textbf{6 Использование информационных ресурсов Internet}}
  $\underset{
    \text{(да, нет, число ссылок в списке литературы)}
  }{
    \underline{\makebox[\remaining][s]{\hfill да, 6 \hfill}}
  }$
}

\restoregeometry

\clearpage

\newgeometry{
  top=20mm,
  right=20mm,
  bottom=20mm,
  left=15mm,
  bindingoffset=0cm
}

\thispagestyle{empty}

{
  \parindent 0pt

  \textbf{7 Использование современных пакетов компьютерных программ и технологий}
  \begin{figure}[h!]
    \centering
    \begin{tabular}{| >{\small\vspace{2pt}}m{10cm} | >{\centering\small\vspace{2pt}}m{3cm} |}
      \toprule
      \centering\textbf{Пакеты компьютерных программ и технологий} & \textbf{Раздел работы} \tabularnewline
      \midrule
      Командный процессор bash & 3, 4 \tabularnewline
      \midrule
      Утилита debootstrap & 3 \tabularnewline
      \midrule
      Менеджер пакетов APT & 3 \tabularnewline
      \midrule
      ПО для развёртывания и управления приложениями в средах с поддержкой контейнеризации Docker & 3 \tabularnewline
      \bottomrule
    \end{tabular}
  \end{figure}\\[-2.5em]

  \textbf{8 Краткая характеристика полученных результатов}
  \uline{В ходе работы над дипломным проектом была спроектирована и реализована система сборки Linux дистрибутивов для встроенных систем. Система сборки была протестирована сотрудниками компании Emlid. Проведена апробация резуль~\-татов работы системы. Созданная система сборки используется в коммерческом проекте Emlid Neutis-n5.\hfill} \\[-1em]

  \titledline{\textbf{9 Полученные гранты, при выполнении работы}}
  $\underset{
    \text{(название гранта)}
  }{
    \underline{\makebox[\remaining][s]{\hfill нет\hfill}}
  }$ \\[-1em]

  \titledline{\textbf{10 Наличие публикаций и выступлений на конференциях по теме выпускной работы}}
  $\underset{
    \text{(да, нет)}
  }{
    \underline{\makebox[\remaining][s]{\hfill нет\hfill}}
  }$ \\[0.5em]
  \titledline{а) 1}
  $\underset{
    \text{(Библиографическое описание публикаций)}
  }{
    \underline{\makebox[\remaining][s]{\hfill}}
  }$ \\[0.5em]
  \titledline{б) 1}
  $\underset{
    \text{(Библиографическое описание выступлений на конференциях)}
  }{
    \underline{\makebox[\remaining][s]{}}
  }$
  \uline{\hfill} \\[-3em]
  \begin{flushright}
    \underline{\makebox[\remaining][s]{}}
  \end{flushright}
  \vskip -0.75em
  \uline{\hfill} \\[-1em]

  Студент
  $\underset{
    \text{\scriptsize (Фамилия, И., О.)}
  }{
    \underline{\makebox[12em][s]{\strut\hfill}}
  }$~~ \signature\\[-0.5em]

  Руководитель
  $\underset{
    \text{\scriptsize (Фамилия, И., О.)}
  }{
    \underline{\makebox[12em][s]{\strut\hfill}}
  }$~~ \signature\\[-0.5em]

  \datetemplate
}

\normalsize
\restoregeometry

\clearpage
