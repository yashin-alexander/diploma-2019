\mysection{РАЗРАБОТКА АРХИТЕКТУРЫ ПРОГРАММНОГО ОБЕСПЕЧЕНИЯ}
\subsection{Обзор работы системы сборки}
\label{subsec:build-system-overview}
Разработка любого программного обеспечения должна начинаться с определения требований, предоставляемых к нему\cite{REQUIREMENTS}.
Для определения функциональных требований необходимо проанализировать процессы, на использование которых направлено разрабатываемое программное обеспечение.
Для определения требований к системе сборки необходимо поэтапно рассмотреть процедуру создания образа.


\begin{enumerate}
  \item Запуск;
  \begin{dashitemize}
    \item Проверка среды исполнения;
    \item Проверка наличия прав суперпользователя;
    \item Считывание конфигурационного файла;
  \end{dashitemize}
  \item Применение настроек из конфигурационного файла;
  \begin{dashitemize}
    \item Экспорт глобальных констант для сборки;
    \item Установка флагов игнорирования стадий сборки;
    \item Установка флагов игнорирования стадий экспорта образа;
  \end{dashitemize}
  \item Сборка образов;
  \begin{dashitemize}
    \item Выполнение prerun-шага;
    \item Выполнение подшагов от 00 до 99;
  \end{dashitemize}
  \item Экспорт образов;
  \begin{dashitemize}
    \item Разметка файла образа;
    \item Установка основных пакетов;
    \item Установка загрузчика;
    \item Установка образа Linux;
    \item Сохранение образа;
  \end{dashitemize}
\end{enumerate}

Для создания общей архитектуры приложения были сформулированы наиболее важные идеи, задачи и требования, предъявляемые к разрабатываемому продукту.

\newpage
\subsection{Требования к реализуемой системе}

Требования к каждой разрабатываемой системе можно разделить на функциональные и нефункциональные\cite{REQUIREMENTS}.
Функциональные требования определяют, требуемое поведение системы в определенных условиях.
Нефункциональные требования определяют свойства или особенности, которым должна обладать система, или ограничение, которое должна соблюдать система.
Другими словами, требования, отличные от функциональных, могут описывать не что система делает, а как хорошо она это делает\cite{REQUIREMENTS}.

К системе сборки были поставлены следующие функциональные требования:
\begin{itemize}
  \item кроссплатформенность;
  \item возможность вручную конфигурировать необходимые предустановленные в образ пакеты;
  \item возможность одновременной сборки нескольких образов: минимального, серверного, расширенного;
  \item наличие загрузчика в образах;
  \item наличие пакетного менеджера в образах;
  \item пригодность для использования образов во время разработки ВС;
\end{itemize}
\newpage
