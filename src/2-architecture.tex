\mysection{РАЗРАБОТКА АРХИТЕКТУРЫ ПРОГРАММНОГО ОБЕСПЕЧЕНИЯ}
\subsection{Обзор работы системы сборки}
\label{subsec:build-system-overview}
Разработка любого программного обеспечения должна начинаться с определения требований, предоставляемых к нему\cite{REQUIREMENTS}.
Для определения функциональных требований необходимо проанализировать процессы, на использование которых направлено разрабатываемое программное обеспечение.
Для определения требований к системе сборки необходимо рассмотреть процедуру создания образа. Рассмотрим эту процедуру поэтапно

Для создания общей архитектуры приложения были сформулированы наиболее важные идеи, задачи и требования, предъявляемые к разрабатываемому продукту.

\newpage
Система сборки должна быть обладать следующими свойствами:
\begin{itemize}
  \item кроссплатформенность;
  \item возможность вручную конфигурировать необходимые предустановленные в образ пакеты;
  \item возможность одновременной сборки нескольких образов: минимального, серверного, расширенного;
  \item наличие загрузчика в образах;
  \item наличие пакетного менеджера в образах;
  \item пригодность для использования образов во время разработки ВС;
\end{itemize}
\newpage
