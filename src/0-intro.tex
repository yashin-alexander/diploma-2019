\mysection*{ВВЕДЕНИЕ}

\textbf{Актуальность темы}. В~настоящее время встраиваемые системы используются во сногих сферах человеческой деятельности. С помощью встраиваемых систем множество сложных процессов способны осуществляться легко, безопасно, вовремя, точно и непрерывно. Такие системы вносят серьезные изменения в повседневную жизнь людей.

Ядро Linux может работать на разных компьютерных архитектурах, большинство из которых довольно популярны во встроенном мире. Все базовые пакеты, позволяющие ОС выполнять основные задачи, подходят для кросс-компиляции, поэтому Linux может быть таким же распространенным, как микроконтроллеры и системы на кристалле (SoC).

\textbf{Цель работы.} Целью данного дипломного проекта является проектирование и реализация системы сборки Linux дистрибутивов для встроенных систем.

\textbf{Задачи работы.} Для достижения поставленной цели необходимо решить следующие задачи:

\begin{itemize}
  \item провести анализ существующих систем сборки 
  \item определить требования к системе сборки 
  \item спроектировать архитектуру системы сборки 
  \item реализовать программное обеспечение системы сборки в соответствии с разработанной архитектурой 
  \item произвести тестирование и апробацию
\end{itemize}

\textbf{Апробация результатов работы}. Наличие документации позволило осуществить открытое тестирование приложения пользователями и, как результат, получить отзывы, сообщения об ошибках и пожелания к функциональности.


Объем и структура работы. 
Работа содержит 999 страниц печатного текста, 999 рисунков, 999 таблицы, список литературы, включающий 999 источников. 
Работа состоит из введения, четырех частей и заключения. 
Во введении обоснована актуальность работы, определены цель и задачи исследования. 
В первой части произведено краткое описание встроенных систем, а также анализ существующих решений, реализующих сборку Linux дистрибутивов. 
Вторая часть содержит описание процесса проектирования архитектуры системы сборки. 
Третья глава описывает реализацию программного обеспечения. 
Четвертая глава описывает тестирование реализованного программного обеспечения. 
В заключении приведены основные результаты работы и возможные направления для дальнейшего развития.
\newpage
